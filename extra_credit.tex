\documentclass[12pt,a4paper,twoside]{report}
\usepackage[margin=1in]{geometry}

\usepackage{xargs}
\usepackage[pdftex,dvipsnames]{xcolor}
\usepackage{setspace}
\doublespacing

\usepackage[utf8]{inputenc}
\usepackage[T1]{fontenc}
\usepackage{textcomp}
\usepackage{amsmath,amssymb,mathtools}


% figure support
\usepackage{import}
\usepackage{xifthen}
\pdfminorversion=7
\usepackage{pdfpages}
\usepackage{transparent}
%\newcommand{\incfig}[1]{%
%	\def\svgwidth{\columnwidth}
%	\import{./figures/}{#1.pdf_tex}
%}

\newcommand{\incfig}[2][1]{%
    \def\svgwidth{#1\columnwidth}
    \import{./figures/}{#2.pdf_tex}
}
\usepackage{float}

\usepackage[none]{hyphenat}

\usepackage{enumerate}


\usepackage{hyperref}

\newcommand\N{\ensuremath{\mathbb{N}}}
\newcommand\R{\ensuremath{\mathbb{R}}}
\newcommand\Z{\ensuremath{\mathbb{Z}}}
\renewcommand\O{\ensuremath{\emptyset}}
\newcommand\Q{\ensuremath{\mathbb{Q}}}
\newcommand\C{\ensuremath{\mathbb{C}}}
\renewcommand\P{\ensuremath{\mathcal{P}}}

\pdfsuppresswarningpagegroup=1



\title{Extra Credit} %document
\author{}
\date{\today} %\today


\begin{document}
\maketitle

\section*{Chapter 3}
\subsection*{Def 3.3.1}
    Let f be a fn. w/ doman $D \subseteq \R$. Then f has a limit as x approaches infinity iff $\exists L \in \R$ s.t. for every $\mathcal{E} > 0, \exists M \in \R^+$ s.t. $|f(x) - L| < \mathcal{E}$, if $x \ge M \text{ and } x \in D$.
    If such an L exists, then L is called the limit of the fn f as x tends to infinity and we write $\lim_{x \to \infty} f(x) = L$

\subsection*{Def 3.1.2}
\end{document}
