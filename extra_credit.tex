\documentclass[12pt,a4paper,twoside]{report}
\usepackage[margin=1in]{geometry}

\usepackage{xargs}
\usepackage[pdftex,dvipsnames]{xcolor}
\usepackage{setspace}
\doublespacing

\usepackage[utf8]{inputenc}
\usepackage[T1]{fontenc}
\usepackage{textcomp}
\usepackage{amsmath,amssymb,mathtools}


% figure support
\usepackage{import}
\usepackage{xifthen}
\pdfminorversion=7
\usepackage{pdfpages}
\usepackage{transparent}
%\newcommand{\incfig}[1]{%
%	\def\svgwidth{\columnwidth}
%	\import{./figures/}{#1.pdf_tex}
%}

\newcommand{\incfig}[2][1]{%
    \def\svgwidth{#1\columnwidth}
    \import{./figures/}{#2.pdf_tex}
}
\usepackage{float}

\usepackage[none]{hyphenat}


\usepackage{hyperref}

\newcommand\N{\ensuremath{\mathbb{N}}}
\newcommand\R{\ensuremath{\mathbb{R}}}
\newcommand\Z{\ensuremath{\mathbb{Z}}}
\renewcommand\O{\ensuremath{\emptyset}}
\newcommand\Q{\ensuremath{\mathbb{Q}}}
\newcommand\C{\ensuremath{\mathbb{C}}}
\renewcommand\P{\ensuremath{\mathcal{P}}}

\pdfsuppresswarningpagegroup=1



\title{Extra Credit} %document
\author{}
\date{\today} %\today


\begin{document}
\maketitle

\section*{Chapter 3}

\subsection*{Def 3.1.1}
    Let f be a fn. w/ doman $D \subseteq \R$. Then f has a limit as x approaches infinity iff $\exists L \in \R$ s.t. for every $\mathcal{E} > 0, \exists M \in \R^+$ s.t. $|f(x) - L| < \mathcal{E}$, if $x \ge M \text{ and } x \in D$.
    If such an L exists, then L is called the limit of the fn f as x tends to infinity and we write $\lim_{x \to \infty} f(x) = L$

\subsection*{Def 3.1.2}
    If $\lim_{x \to \infty} f (x) = L$, then the line $y = L$ is called a horizontal asymptote for the function f.
\subsection*{Thm 3.1.6}
    Suppose that $D \subseteq \R$ is an unbounded above domain of the function f ; that is, $D$ contains arbitrarily large values. Then, $\lim_{x \to \infty} f (x) = L$ iff for every sequence $\{ x_n \}$ in $D$ that diverges to plus infinity, that is, $\lim_{n \to \infty} x_n = \infty$, the sequence $\{ f (x_n) \}$ converges to $L$.

\subsection*{Thm 3.1.7}
    Suppose that the functions f, g, and h are defined on $D \subseteq \R$, which is unbounded above, with $\lim_{x \to \infty} f (x) = A, \lim_{x \to \infty} g (x) = B$, and $\lim_{x \to \infty}  h (x) = C$. Then
    \begin{enumerate}[(a)]
        \item $\lim_{x \to \infty} f(x)$ is unique
        \item f must be eventually bounded above and below
        \item $\lim_{x \to \infty} [f(x) - A] = 0$
        \item $\lim_{x \to \infty} |f(x)| = | \lim_{x \to \infty} f(x)| = |A|$
        \item $\lim_{x \to \infty} ( f \pm g )(x) = \lim_{x \to \infty} f(x) \pm \lim_{x \to \infty} g(x) = A \pm B$
        \item $\lim_{x \to \infty} ( f g )(x) = [ \lim_{x \to \infty} f(x) ] [ \lim_{x \to \infty} g(x) ]= A B$
        \item $\lim_{x \to \infty} [f(x)]^{n} = [\lim_{x \to \infty} f(x)]^{n} = A^{n}, \forall n \in \N$
	\item $\lim_{x \to \infty} \left( f/g \right) (x) = \dfrac{A}{B} \ if B \not= 0$
	\item $\lim_{x \to \infty} \sqrt[n]{f(x)} = \sqrt[n]{\lim_{x \to \infty} f(x)} = \sqrt[n]{A}$ if $A \ge 0$ \& $f(x) \ge 0 \ \forall x \in D$, with $n \in \N$
	\item $A \le B \ if \ f(x) \le g(x)$ eventually for $x \in D$
	\item $A \le B \le C$ if $f (x) \le g (x) \le h (x)$ eventually for $x \in D$. This property is called the sandwich (or squeeze) theorem
 \end{enumerate}

\subsection*{Thm 3.1.8}
    If the function $f$ is defined on an unbounded above domain $D \subseteq \R$ and is eventually monotone and eventually bounded, then $\lim_{x \to \infty} f (x)$ is finite.

\subsection*{Def 3.1.9}
    Let f be a function with domain $D \subseteq \R$, which contains arbitrarily large values. We say that $f$ tends to plus infinity as x tends to $+ \infty$ iff for any real $K > 0$, there exists a real number $M > 0$ such that $f (x) > K$ provided that $x \ge M$ and $x \in  D$. Whenever this is the case, we write $\lim_{x \to \infty} f (x) = +\infty$

\subsection*{Def 3.1.10}
    Let $f$ be a function with domain $D \subseteq \R$, which contains arbitrarily large negative values. Then $\lim_{x \to -\infty} f (x) = L$ iff for every $\epsilon > 0$ there exists a real number $M > 0$ such that $|f (x) - L| < \epsilon$ if $x \le M$ and $x \in D$

\subsection*{Def 3.2.1}
    Suppose that a function $f : D \to \R$, and suppose that $a$ is an accumulation point of $D$. The function $f$ has a limit as $x$ approaches (or as $x$ tends to) $a$ iff there exists a real number $L$ such that for every $\epsilon > 0$ there exists a real number $\delta > 0$ such that
    \[
    |f(x) - L| < \epsilon \text{ provided that } 0 < |x - a| < \delta \text{ and } x \in D
    \] 
    write $\lim_{x \to a} f(x) = L$

\subsection*{Thm 3.2.5}
    Suppose that functions $f, g, h : D \to \R$, with $D \subseteq \R$, $a$ is an accumulation point of $D$,$\lim_{x \to a} f(x) = A$,$\lim_{x \to a} g (x) = B$, and $\lim_{x \to a} h (x) = C$. Then all of the conclusions for Theorem 3.1.7 are true with $\infty$ replaced by $a$ and with "eventually" replaced by "near $x = a$."

\subsection*{Thm 3.2.6}
    Let the function $f$ be defined on some deleted neighborhood $D$ of the real number $a$. The following two statements are equivalent.
    \begin{enumerate}[(a)]
        \item $\lim_{x \to a} f(x) = L$ 
	\item For every sequence $ \{ x_n \} $ converging to $x = a$, with $x_n \in D$ and $x_n \not= a$ eventually, the sequence $\{ f(x_n) \} $ converges to $L$
    \end{enumerate}

\subsection*{Def 3.2.12}
    Suppose that the function $f : D \to \R $ with $D$ a subset of $\R$ and $a$ an accumulation point of $D$. Then the function $f$ tends to plus infinity as $x$ approaches, tends to, $a$ iff for any given real number $K > 0$, there exists $\delta > 0$ such that $f (x) > K$, provided that $0 < |x - a| < \delta$ and $x \in D$. Write $\lim_{x \to a} f (x) = +\infty$
\subsection*{Thm 3.2.14}
    Let the functions $f$ and $g$ be defined on some deleted neighborhood  of $x = a$. If $\lim_{x \to a} f(x) = L > 0$ and $\lim_{x \to a} g(x) = + \infty$, then $\lim_{x \to a} (fg) (x) = + \infty$.

\subsection*{Def 3.3.1}
    Suppose that the function $f : D \to \R$, with $D$ a subset of $\R$ and $a$ an accumulation point of the set $D \cap (a, \infty) = \{x \in  D | x > a \}$ Then the function $f$ has a right-hand limit (limit from the right) as $x$ approaches, tends to, $a$ iff there exists a real number $L$ such that for every $\epsilon > 0$ there exists a positive real number $\delta > 0$ such that
    \[
    |f(x) - L| < \epsilon \text{ provided that } 0 < x - a < \delta \text{ and } x \in D 
    \] 
    we write $ \lim_{x \to a^{+}} f(x) = L$

\subsection*{Def 3.3.2}
Suppose that the function $f : D \to \R$, with $D$ a subset of and $a$ an accumulation point of $D \cap (a, \infty)$. Then the function $f$ tends to infinity as $x$ approaches, tends to, $a$ from the right iff for any given real number $K > 0$, there exists a positive $\delta> 0$ such that $f (x) > K$, provided that $0 < x - a < \delta$ and $x \in D$. We write $\lim_{x \to a^{+}} f(x) = +\infty$

\subsection*{Def 3.3.4}
    If the limit from the right or from the left at $x = a$ of a function $f$ is infinite, meaning $+\infty$or $-\infty$, then the line $x = a$ is called a vertical asymptote.

\subsection*{Thm 3.3.7}
    Let a function $f$ be defined for $x \in (0, a)$, with $a > 0$ a real number. If
    \[
    \lim_{x \to 0^{+}} f(x) \text{ or } \lim_{t \to \infty} f\left( \frac{1}{t} \right)
    \] 

\section*{Chapter 4}

\subsection*{Def 4.1.1}

\subsection*{Def 4.1.2}

\subsection*{Sequential Criterion for Continuity Thm}

\subsection*{Def 4.1.6}

\subsection*{Thm 4.1.7}

\subsection*{Thm 4.1.8}

\subsection*{Thm 4.1.9}

\subsection*{Def 4.2.1}

\subsection*{Def 4.2.3}

\subsection*{Def 4.2.5}

\subsection*{Def 4.2.7}

\subsection*{Def 4.3.1}

\subsection*{Def 4.3.2}

\subsection*{Thm 4.3.3}

\subsection*{Thm 4.3.4}

\subsection*{Thm 4.3.5}

\subsection*{Thm 4.3.6}

\subsection*{Def 4.3.7}

\subsection*{Cor 4.3.8}

\subsection*{Cor 4.3.9}

\subsection*{Thm 4.3.10}

\subsection*{Thm 4.3.11}

\subsection*{Def 4.4.2}

\subsection*{Def 4.4.1}

\subsection*{Def 4.4.3}

\subsection*{Thm 4.4.6}

\subsection*{Def 4.4.6}

\subsection*{Thm 4.4.11}

\subsection*{Thm}
\end{document}

